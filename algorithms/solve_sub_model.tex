\begin{algorithm}[H]
    \caption{Solve Submodel}\label{algo:solve-submodel}
    \begin{algorithmic}[1]
        \REQUIRE $submodel: \text{Submodel from \Cref{algo:submodel-build}},\ time: \text{Time limit}$
        \STATE {$lin\_model = MIPSolver::NewLP(submodel,\ time);$}
        \STATE {$lin\_solution,\ status = lin\_model.solve(); $}
        \IF {$ status\ \textbf{is}\ OPTIMAL $}
            \STATE $int\_model = MIPSolver::NewMIP(submodel,\ time);$
            \STATE $int\_model.solution\_count(1);$
            \STATE $int\_sol, status = lin\_model.solve();$
            \IF {$ status\ \textbf{is}\ INFEASIBLE $}
                \RETURN {$lin\_solution,  LIN\_FEASIBLE$}    
            \ELSIF{$ status\ \textbf{id}\ TIMEOUT $}
                \RETURN {$\emptyset, TIMEOUT $}
            \ELSE 
                \RETURN {$lin\_solution,  INT\_FEASIBLE$}    
            \ENDIF
        \ELSE
            \RETURN {$\emptyset,  INFEASIBLE$}
        \ENDIF
    \end{algorithmic}
\end{algorithm}

\paragraph{Description} \Cref{algo:solve-submodel} requires as input a valid submodel, from \Cref{algo:submodel-build} and the current time limit. The module $MIPSolver$
is an abstraction of a general purpose MIP soler, it has two methods:
\begin{itemize}
    \item NewLP : create a \emph{Model} object from the sub model, with the specified time limit. All integer constraints are ignored.
    \item NewMIP: create a \emph{Model} object from the sub model, with the specified time limit. All integer constraints are preserved.
\end{itemize}

This \emph{Model} object has the \emph{solve} method. This method solves the problems accordind to the model configuration, for example time limit or solution count limit. 
\emph{solve} return two values: the first one is the model solution (i.e. a \href{https://en.wikipedia.org/wiki/Associative_array}{dictionary}), 
the second is the optimization
status. The possible output\footnote{The actual mip solver may support more than this statuses, but those are the strictly required to implement Feature Kernel } statuses are: 
\begin{enumerate}
    \item OPTIMAL: the optimization ended with a proven optimal result,
    \item INFEASIBLE: the model is infeasible,
    \item TIMEOUT: the optimization took more then the given timeout
    \item SOLUTION\_COUNT: the optimization found the maximun number of feasbile solution, but is not possible to prove that this solution is optimal.
\end{enumerate}

Notice that the statuses from the MIP solver and the statuses returned by \Cref{algo:solve-submodel} are not the same: statuses from \Cref{algo:solve-submodel} can be 
considered as inner constants. 

The possible outcomes are straighforeward as described in \Cref{sec:submodel-result}, according to the optimization status \Cref{algo:solve-submodel} returns the current
solution (if available) and the appropriate status. 