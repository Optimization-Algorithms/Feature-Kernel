\begin{algorithm}[H]
    \caption[FAKE TITLE]{Feature Kernel}\label{algo:feature-kernel}
    \begin{algorithmic}[1]
        \REQUIRE $inst: \text{MIP instance},\ config: \text{Configuration}$
        \STATE {$solutions = build\_solutions^{\ref{algo:build-solutions}}(inst,\ config);$} \label{fk:line:init}
        \STATE {$dataset = generate\_data\_set(solutions);$}    
        \STATE {$importance = generate\_feature\_importance(dataset); $} \label{fk:line:feature-importance}
        \STATE {$kernel,\ bucket\_vars = build\_kernel(solutions,\ importance,\ config);$}
        \RETURN {$ (kernl\, bucket\_vars); $}
    \end{algorithmic}
\end{algorithm}

\paragraph{Description} \Cref{algo:feature-kernel} shows an high-level description of the method. On line \ref{fk:line:init} 
\Cref{algo:feature-kernel} builds the initial solution set, see \Cref{algo:build-solutions} for the details. What metters here is
that there is a list of submodels, with relative variable values and feasibility status if the model was at least continuous feasible. This 
set can be used to generate the dataset, train the Random Forrest and build the feature importance vector (on line \ref{fk:line:feature-importance})


