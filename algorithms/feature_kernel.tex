\begin{algorithm}[H]
    \caption{Feature Kernel}\label{algo:feature-kernel}
    \begin{algorithmic}[1]
        \REQUIRE $inst: \text{MIP instance},\ config: \text{Configuration}$
        \STATE {$solutions = build\_solutions^{\ref{algo:build-solutions}}(inst,\ config);$} \label{fk:line:init}
        \STATE {$dataset = generate\_data\_set(solutions);$}    
        \STATE {$importance = generate\_feature\_importance(dataset); $} \label{fk:line:feature-importance}
        \STATE {$kernel,\ bucket\_vars = build\_kernel(solutions,\ importance,\ config);$} \label{fk:line:build-kernel}
        \RETURN {$ (kernl,\ bucket\_vars); $}
    \end{algorithmic}
\end{algorithm}

\paragraph{Description} \Cref{algo:feature-kernel} shows an high-level description of the method. On line \ref{fk:line:init} 
\Cref{algo:feature-kernel} builds the initial solution set, see \Cref{algo:build-solutions} for the details. What meters here is
that there is a list of sub models, with relative variable values and feasibility status if the model was at least continuous feasible. This 
set can be used to generate the dataset, train the Random Forrest and build the feature importance vector (on line \ref{fk:line:feature-importance}).
The feature importance vector is then used to split variables into  kernel variables and bucket variables. Function $build\_kernel$, on line \ref{fk:line:build-kernel},
builds the initial kernel but does not build the initial buckets, this is done in a separate section of the program. \Cref{algo:feature-kernel} simply implements a 
new Kernel builder, so it can be used in substitution of the default Kernel builder.



