\begin{algorithm}[H]
    \caption[Generate Random Sub model]{Generate Random Sub model\footnotemark}\label{algo:submodel-build}
    
    \begin{algorithmic}[1]
        \REQUIRE $model : \text{Valid Model Instance}, ur : \text{Usage Ratio}$
        \IF{$ur == 1$}
            \RETURN
        \ENDIF
        \STATE $count = (model.len() \times ur)\ \textbf{as integer};$
        \FOR{$var\ \textbf{in}\ model.vars() \footnotemark $}
            
            \STATE {$var.disable();$}
        \ENDFOR
        \FOR  {$ i\ \textbf{in}\  unique\_random\_numbers(count,\ model.len()) $} 
            \STATE {$ var = model.get\_var\_by\_index(i); $}
            \STATE {$ var.enable(); $}
        \ENDFOR
    \end{algorithmic}
\end{algorithm}

\footnotetext[3]{For the actual implementation check for function \texttt{generate\_random\_sub\_model} in the source file \path{ks.py/ks_engine/feature_kernel.py}}
\footnotetext{\textbf{for} loop in this pseudocode are foreach loops, see \href{https://en.wikipedia.org/wiki/Foreach_loop}{here}}


\paragraph*{Description}
    \Cref{algo:submodel-build} starts by checking if $ur$ is $1$, in this case the submodel is indeed the full model,
    so no further actions are required. In the normal case this should not 
    happen. Once done this simple check \Cref{algo:submodel-build} computes the number of
    variables in the submodel (stored in the variable $count$)  the associated function 
    $len$ returns the length of the model (the number of variables in it), the multiplication produces a floating point result, for the purpose of \Cref{algo:submodel-build}
    $count$ must be an integer number, this is correction done by the cast at end of the line. Now \Cref{algo:submodel-build} disable all the variables in the model. This step is
    done by the first for loop using the associated function $disable$. $disable$ is usually implemented by adding a new constraint to the model that
    forces the variable to zero. At this point the submodel can be build by sampling the variables. Selected variable will be reactivated using the associated function
    $enable$. $enable$ can be implemented to simply remove the constraint added by $disable$.
    The function $unique\_random\_numbers$ gets to arguments: the first tells how many number should the produced list contain, the second tells the upper limit for the output values,
    the lower is implicitly set to zero. So this function produces a list of $count$ unique random numbers in range $\left[0, model.len()\right[$ so they can be safely used as indexes. 

