\begin{algorithm}[H]
    \caption{Build Solutions}\label{algo:build-solutions}
    \begin{algorithmic}[1]
        \REQUIRE $inst: \text{MIP instance}, config: \text{Configuration}$
        \STATE {$ur = 0.01;$} \refcomment{sec:usage-ratio}
        \STATE {$solutions = empty\_list();$} \label{build-solutions:line:empty-list}
        \STATE {$time\_out = config.min\_time; $}
        \FOR {$  i\ \textbf{in}\  count(config.count)$}
            \STATE {$ submodel = generate\_random\_submodel(inst,\ ur)^{\ref{algo:submodel-build}};$}
            \STATE {$ solution, status = solve\_submodel^{\ref{algo:solve-submodel}}(submodel, time\_out);  $}
            \IF {$ status\ \textbf{is}\ INFEASIBLE $}
                \STATE $ur = \sqrt[5]{ur^4};$ \refcomment{eq:ur-infease-grow}
            \ELSIF {$status\ \textbf{is}\ LIN\_FEASIBLE$}
                \STATE $ur = 1.1 \times ur;$ \refcomment{eq:ur-lin-fease-grow}
                \STATE $insert\_solution(solutions, solution, status);$
            \ELSIF {$status\ \textbf{is}\ INT\_FEASIBLE$}
                \STATE $ur = 0.9 \times ur;$ \refcomment{eq:ur-fease-grow}
                \STATE $insert\_solution(solutions, solution, status);$
            \ELSIF {$status\ \textbf{is}\ TIMEOUT$}
                \IF {$ time\_out < config.max\_time $}
                    \STATE {$ time\_out = time\_out + 1;$}
                \ENDIF
            \ENDIF
        \ENDFOR
        \RETURN $solutions$
    \end{algorithmic}
\end{algorithm}

\paragraph{Description} \Cref{algo:build-solutions} is a core component of Feature Kernel. It solve any submodel to understand its feasibility status and then controls 
the two main inner parameter of the method: $Usage\ Ratio$, see \Cref{eq:usage-ratio}, and timeout, see \Cref{sec:timeout}.

\Cref{algo:build-solutions} requires as input some kind of MIP instance and a configuration. The configuration is just a collection of named values 
(i.e. a \href{https://en.wikipedia.org/wiki/Struct_(C_programming_language)}{C Struct} or a 
\href{https://docs.python.org/3/library/collections.html#namedtuple-factory-function-for-tuples-with-named-fields}{Python namedtuple}) 
so it is possible to pass any parameter with just one name. Config contains any useful configuration parameter, like time limits, submodel count and the split 
criterion. 

On line \ref{build-solutions:line:empty-list} the function $empty\_list$ create a new empty dynamic vector 
(i.e. a \href{https://en.wikipedia.org/wiki/List_(abstract_data_type)}{list}) of the appropriate type to store solutions. This list can be used as the first argument of 
the $insert\_solution$ that under the hood creates a new record and append it as the end of the list. 

Then \Cref{algo:build-solutions} generate and solve the given number of sub models, end for each submodel it check the output status and act accordingly:
\begin{itemize}
    \item If the model is feasible it is stored and $ur$ is update as described in \Cref{eq:ur-fease-grow} and \Cref{eq:ur-lin-fease-grow}
    \item If the model is infeasible the solution is empty, so it is ignored, and $ur$ is update as described in \Cref{eq:ur-infease-grow}
    \item If the model ends with a timeout the solution is empty, so it is ignored, and $timeout$ is updated as described in \Cref{sec:timeout}
\end{itemize}

Once all the submodel have been created and tested \Cref{algo:build-solutions} returns the list of valid solutions.
