\section{Main Idea}
Feature Kernel  is a method to generate an initial kernel for Kernel Search. 
It aims to avoid the reliance on the linear relaxation to build the initial 
kernel. Although a kernel built on top of the base variables in the linear relaxation can provide in most
cases a good staring point, this method also presents some issues:
\begin{enumerate}
    \item Some problems are also difficult to solve when are linearly relaxed
    \item The linear relaxation does provide little to none information about the integer model
    \item In some models (i.e. knapsack or tsp based ones) the linear solution is very different from 
    the integer one, so this initial kernel is not particularly significant 
\end{enumerate}

There is also a big limitation with linear relaxation in general: the solver can hardly scale to a multi threading execution, so 
if the linear relaxation particularly  hard it will always take a long time to solve independently from the machine used to solve it.

Feature Kernel aims to compute a ranking for each variable in the problem and then, using this ranking, separate kernel from bucket variables. Once 
done that Kernel Search can proceed as normal. The ranking is compute by randomly build a number of sub problems then for each sub problem finds if it
is infeasible, linearly feasible or continuously feasible. Each of those states will associate a specific class to each sub problem. Once each sub model status
has been found all the sub models are used as datasets for a Random Forrest Classifier, a Machine Learning technique. Once the Random Forrest is trained on
the given datasets it produces a \emph{Feature Importance Vector}\footnote{This is where the name Feature Kernel came from}, a ranking of the variables importance into determining the final result. 
This ranking is then used to sort the variables. Variables with an high ranking are used set into the initial kernel, the others are used to build buckets.

This method aims to provide a series of advantages:
\begin{enumerate}
    \item Better understand which variables are important to build a feasible solution, allowing Kernel Search to build a better initial kernel and a better buckets.
    \item Avoid to remain stuck into a complex linear relaxation problem.
    \item Allow the user to understand a good strategy to solve the problem with Kernel Search or to understand if Kernel Search is the right technique to solve the problem.
    \item Allow for a better usage of multi threading during the initial kernel build, both on a single machine or within a cluster
\end{enumerate}


It is important to understand that Feature Kernel simply changes the way the initial kernel and the ranking of bucket variables are computed, anything else stays the same. So Kernel Search
will not change in anyway how bucket variables are sorted and how they are groupped into the actual buckets. 




