\section{Experimental Results}

In general \fk{} behaves very poorly. All the fallowing problems, taken from MIP lib, have been solved 
using the method described above. To be more concise we  present only the results from \fk{}, because 
the rest of the \ks{} is the same as in a standard implementation. 

In order to find some patterns and to identify the actual quality of the method all the following tests was
run with the same configuration:
\begin{table}[H]
    \centering
    \begin{tabular}{|l|l|}
        \hline
        Parameter & Value \\ \hline\hline
        Count & 50 \\ \hline
        Min Time & 10 \\ \hline
        Max Time & 40 \\ \hline
    \end{tabular}
    \caption{\fk{} configuration}\label{tab:ks-config}
\end{table}

In this configuration are not present all the required parameters: the remaining are omitted because they are not 
involved into the way \fk{} works.

In the following examples we use a simple schema to show the name of the instance, its URL and a graph showing 
the convergence of the method. In the abscissa there is the \fk{} iteration, on the ordinate there is the \emph{usage ratio} (see \Cref{eq:usage-ratio}).
A \fk{} iteration can end in four different statuses (see \Cref{sec:submodel-result}), identified on each graph using four different colors, as described:
\begin{table}[H]
    \centering
    \begin{tabular}{|l|l|}
        \hline
        Status & Color \\ \hline\hline
        Infeasible & Red \\ \hline
        Linear Feasible & Blue \\ \hline
        Integer Feasible & Green \\ \hline
        Timeout & Black \\ \hline
    \end{tabular}
    \caption{Color legend for feasibility convergence graphs}
\end{table}

\newcommand{\Figure}{}
\newcommand{\Table}{}
\newcommand{\Instance}{}

\newcommand{\importExample}[2]{
    \begin{table}[H]
        \centering
        \begin{tabularx}{\textwidth}{l X}
            Name: & #1 \\ 
            URL:  & \url{#2} 
        \end{tabularx}
    \end{table}
    \begin{figure}[H]
        \graphicspath{{test-results/}}
        \centering
        \includegraphics[width=0.95\textwidth, keepaspectratio]{#1-ratio}
        \caption{Feasibility convergence for #1}\label{fig:conv-#1}
        \graphicspath{{images/}}
    \end{figure}
    \input{test-results/#1-ratio}\unskip
    \renewcommand{\Figure}{\Cref{fig:conv-#1}}
    \renewcommand{\Table}{\Cref{tab:stats-#1}}
    \renewcommand{\Instance}{#1}
}
\importExample{air03}{https://miplib.zib.de/instance_details_air03.html}

In this first example one can see that \fk{} starts finding and integer feasible solution
only when the $ur > 0.851$. Also is not \fk{} has not been able to find a linear feasible model 
until $ur < 0.768$. From the \Figure{} one can also notice that the major part of the generated 
models are infeasible, so they are completely useless to identify information about the importance 
of each variable. From \Table{} it is possible to see that $ur$ does not define a precise border between 
Infeasible and Linear Feasible region or between Linear and Integer Feasible region.


\importExample{beavma}{https://miplib.zib.de/instance_details_beavma.html}

In \Instance{} the situation is a bit different: \fk{} has not been able to find a Linear Feasible model. 
Besides that also in this case reading carefully \Table{} allows to understand that $ur$ does not define any 
boarder. From \Figure{} one can see that \fk{} generated on 3 valid sub-models for \Instance{}, and all with a $ur > 0.969$. 
This means that the value of the score computed by \fk{} is almost insignificant. An insignificant dataset will produce a 
meaningless random forest that will bring to a meaningless kernel. 


\importExample{mas74}{https://miplib.zib.de/instance_details_mas74.html}

\Instance{} is a lucky case: it is the only where \fk{} has been able to find an Integer Feasible model with a small $ur$. 
This is the desired behavior of the method: continuous jumps between feasible and infeasible models in order to 
find with variables are more \emph{important}. It is worth notice, from \Table{}, that the submodel never reached a maximum
$ur = 0.615$ so a large part of all possible results was not controlled by \fk{}. It is not possible know if exploring 
the rest of the \emph{ur space} could allow to find a better initial kernel. 

\importExample{neos-2294525-abba}{https://miplib.zib.de/instance_details_neos-2294525-abba.html}

\Instance{}, in contrast of the other instances described above, is considered \emph{hard}. This fact becomes evident when one
takes a close look to \Figure{}: there are a lot of black dots showing that a good number of instances was stopped due to  timeout and
not because they were infeasible. With \Instance{} \fk{} was able to find a clear boarder between infeasible and linear feasible 
instances: this is probably consequence of the structure of the constraints. Although it could be possible to increase the 
maximum time allowed to solve an instance, it could be counterproductive infect the \fk's goal is to reduce the time spent 
building an initial kernel. 


\newcommand{\SM}{\aleph^{m}}
\subsection{Why \fk{} performes so poorly}
For semplicty let us identify with $I$ the original instance, with  $\SM$ the set of 
all the subinstances with $m$ variables enabled and with $\SM_{i}$ the $i$-th instance 
with $m$ variables enabled. We also define $\phi^{m} \subseteq \SM $ as the set of feasible instances
in $\SM$. 
It is implied that $m \leq n$.

At each iteration \fk{} fixes $\SM$ (where $m = \lfloor ur * n \rfloor$) and randomly picks an
instance $\SM_{i}$. Of course when $m$ is small (i.e. $ur \approx 0.2$) is very unlikely that any 
model in $\SM$ is feasible. 
It also easy to immagine that when $m$ is large (i.e. $ur \approx 0.9$) is very unlikely
that there is not any feasible instances in $\SM$. It is correct to say that all linear model have a minimal $ur$ below that 
the number of enabled variable is not enought to satisfy the constraints of the problem. Let be
\begin{align*}
    \tilde{m} = \min m \mid \SM{}\ \text{contains a feasible instance} \label{eq:min-m}
\end{align*}
This means that any information obtained by \fk{} when $ur < \frac{m}{n}$ is completely meaningless: it is obvious 
that the problem will be infeasible and all the selected variables will be penalized by the method, but it is incorrect 
to infer this information because the model cannot be feasible. 
Unfortunatly there are not easy ways to identify the actual $\tilde{m}$ so we are stuck into the guessing performed by 
\fk{}.


Given a set of $n$ elements the number of subsets with $m$ elements is the binomial coefficent
of $n$ over $m$, so the cardinality of $\SM$ is
\begin{align}
    |\SM| = \binom{n}{m}
\end{align}

\begin{figure}[ht]
    \centering
    \includegraphics[keepaspectratio, width=\textwidth]{binomial-20}
    \caption{$|\SM|$ when $n = 20$}\label{fig:plot-binom-20}
\end{figure}

As one can see from \Cref{fig:plot-binom-20} $|\SM|$ is a large number even when $n$ is relatively small. 
In general the number of feasible instance in $\SM$ (with $m > \tilde{m}$) is small so the probably of 
randomly peaking a good $\SM_{i}$ is
\begin{align}
    \frac{|\phi^{m}|}{|\SM|}
\end{align}

In the applications this is obvious that $|\phi^{m}| \ll |\SM|$, in general $|\phi^{m}|$ is so small with respect to $|\SM|$
that it is safe to assume
\begin{align}
    \frac{|\phi^{m}|}{|\SM|} \approx \binom{n}{m}^{-1} \label{eq:prob-good-instance}
\end{align}

\Cref{eq:min-m} show us why \fk{} has so many issues in the initial phase while
\Cref{eq:prob-good-instance} shows why for some instances was so difficult to find feasbile sub instances until $ur$ was close to $1$. 
In the beginnig simply there are no feasible instances to find, but \fk{} looks for them anyway. In the middle phase \fk{} must be 
extremely lucky to find a feasible instance. Only in the final phase, with $m$ close to $n$ and $|\SM|$ reasonably small, \fk{} was able 
to find a good instance by peaking a random one. \fk{} requires a good amount of luck in order to perform properly and in practice 
this is not possible.

There are instances, like mas74 which is a knapsack problem, where $|\phi^{m}|$ is large and in this cases \fk{} could perform well. This are 
cases where due to the form of the constraints it is hard to generate an infeasible submodel. Other instances, like neos-2294525-abba  which is 
a set patitioning problem, have a very small $|\phi^{m}|$ and so \fk{} cannot perform well. This are instances where also with the full model
it is hard to find a solution.








