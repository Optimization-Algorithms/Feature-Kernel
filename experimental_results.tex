\section{Experimental Results}

In general \fk{} behaves very poorly. All the fallowing problems, taken from MIP lib, have been solved 
using the method described above. To be more concise we  present only the results from \fk{}, because 
the rest of the \ks{} is the same as in a standard implementation. 

In order to find some patterns and to identify the actual quality of the method all the following tests was
run with the same configuration:
\begin{table}[H]
    \centering
    \begin{tabular}{|l|l|}
        \hline
        Parameter & Value \\ \hline\hline
        Count & 50 \\ \hline
        Min Time & 10 \\ \hline
        Max Time & 40 \\ \hline
    \end{tabular}
    \caption{\fk{} configuration}\label{tab:ks-config}
\end{table}

In this configuration are not present all the required parameters: the remaining are omitted because they are not 
involved into the way \fk{} works.

In the following examples we use a simple schema to show the name of the instance, its URL and a graph showing 
the convergence of the method. In the abscissa there is the \fk{} iteration, on the ordinate there is the \emph{usage ratio} (see \Cref{eq:usage-ratio}).
A \fk{} iteration can end in four different statuses (see \Cref{sec:submodel-result}), identified on each graph using four different colors, as described:
\begin{table}[H]
    \centering
    \begin{tabular}{|l|l|}
        \hline
        Status & Color \\ \hline\hline
        Infeasible & Red \\ \hline
        Linear Feasible & Blue \\ \hline
        Integer Feasible & Green \\ \hline
        Timeout & Black \\ \hline
    \end{tabular}
    \caption{Color legend for feasibility convergence graphs}
\end{table}

\newcommand{\Figure}{}
\newcommand{\Table}{}
\newcommand{\Instance}{}

\newcommand{\importExample}[2]{
    \begin{table}[H]
        \centering
        \begin{tabularx}{\textwidth}{l X}
            Name: & #1 \\ 
            URL:  & \url{#2} 
        \end{tabularx}
    \end{table}
    \begin{figure}[H]
        \graphicspath{{test-results/}}
        \centering
        \includegraphics[width=0.95\textwidth, keepaspectratio]{#1-ratio}
        \caption{Feasibility convergence for #1}\label{fig:conv-#1}
        \graphicspath{{images/}}
    \end{figure}
    \input{test-results/#1-ratio}\unskip
    \renewcommand{\Figure}{\Cref{fig:conv-#1}}
    \renewcommand{\Table}{\Cref{tab:stats-#1}}
    \renewcommand{\Instance}{#1}
}
\importExample{air03}{https://miplib.zib.de/instance_details_air03.html}

In this first example one can see that \fk{} starts finding and integer feasible solution
only when the $ur > 0.851$. Also is not \fk{} has not been able to find a linear feasible model 
until $ur < 0.768$. From the \Figure{} one can also notice that the major part of the generated 
models are infeasible, so they are completely useless to identify information about the importance 
of each variable. From \Table{} it is possible to see that $ur$ does not define a precise border between 
Infeasible and Linear Feasible region or between Linear and Integer Feasible region.


\importExample{beavma}{https://miplib.zib.de/instance_details_beavma.html}

In \Instance{} the situation is a bit different: \fk{} has not been able to find a Linear Feasible model. 
Besides that also in this case reading carefully \Table{} allows to understand that $ur$ does not define any 
boarder. From \Figure{} one can see that \fk{} generated on 3 valid sub-models for \Instance{}, and all with a $ur > 0.969$. 
This means that the value of the score computed by \fk{} is almost insignificant. An insignificant dataset will produce a 
meaningless random forest that will bring to a meaningless kernel. 


\importExample{mas74}{https://miplib.zib.de/instance_details_mas74.html}

\Instance{} is a lucky case: it is the only where \fk{} has been able to find an Integer Feasible model with a small $ur$. 
This is the desired behavior of the method: continuous jumps between feasible and infeasible models in order to 
find with variables are more \emph{important}. It is worth notice, from \Table{}, that the submodel never reached a maximum
$ur = 0.615$ so a large part of all possible results was not controlled by \fk{}. It is not possible know if exploring 
the rest of the \emph{ur space} could allow to find a better initial kernel. 

\importExample{neos-2294525-abba}{https://miplib.zib.de/instance_details_neos-2294525-abba.html}

\Instance{}, in contrast of the other instances described above, is considered \emph{hard}. This fact becomes evident when one
takes a close look to \Figure{}: there are a lot of black dots showing that a good number of instances was stopped due to  timeout and
not because they were infeasible. With \Instance{} \fk{} was able to find a clear boarder between infeasible and linear feasible 
instances: this is probably consequence of the structure of the constraints. Although it could be possible to increase the 
maximum time allowed to solve an instance, it could be counterproductive infect the \fk's goal is to reduce the time spent 
building an initial kernel. 








